\documentclass[addpoints]{exam}
\pagestyle{headandfoot}
\usepackage{amsmath, amsfonts}
\usepackage{verbatim}
\usepackage{graphicx}
\usepackage[usenames,dvipsnames]{color}
\usepackage[utf8]{inputenc}
\usepackage{enumitem}
\usepackage{amsmath,amsfonts,amssymb,amsthm}
\usepackage{parskip}
\usepackage{caption}
\usepackage{subcaption}

\newcommand{\semester}{WS 2022/2023}
\runningheader{\students}{}{\semester} 
\runningfooter{}{\thepage}{} 
\headrule
\footrule

% ---------- Modify team name, students, exercise number here ----------
\newcommand{\students}{Arpit Baranwall, Ali Sarlak, Natik, Mohamadreza Sharifi}

% ---------- End Modify ----------

\title{Embedded Systems Entrepreneurship}
\author{Team: \students}
\date{\today}
 
\begin{document}
\maketitle

% ---------- Add Solution below here ----------
% ------------------------1----------------------------------
% ------------------------1-a----------------------------------
\section{Introduction}
Our research has identified a gap in the medical market pertaining to the
availability of a device capable of quantifying the masticatory patterns exhibited by patients
when consuming various types of food. This device would prove beneficial in conjunction with expert analysis
, aiding in the diagnosis of potential anomalies in patients. The process of mastication in humans holds
significant importance across multiple domains, particularly within dentistry and the digestive system.
Proper chewing of food is crucial for optimal digestive functioning, as inadequate mastication can lead
to complications within the digestive system and potentially contribute to related ailments.
Additionally, through the examination of collected data, dentists can identify any irregularities in
the jaws and teeth, which may be indicative of suboptimal chewing patterns.

In response to the identified demand for such a device, a novel pair of glasses has been developed (Figure \ref{fig:glasses}),
incorporating specialized sensors that possess a high sensitivity to resonance.
These sensors enable the accurate capture of jaw movements during the process of mastication for various
food types. Notably, these sensors have the capability to detect and record a wide range of jaw movements,
encompassing activities such as speaking, chewing, and other facial muscle activities. Consequently,
postprocessing of the collected data, specifically signal processing, is imperative to isolate and extract the relevant
portion of the signal corresponding to the chewing of food. Furthermore, the project's objective is twofold: firstly,
to identify the specific segment of the recorded signal that corresponds to the patient's chewing activity, and secondly
, to discern the type of food being chewed based on the characteristic chewing patterns exhibited in the signal.

\begin{figure}
    \centering
    \includegraphics[width=0.5\textwidth]{img/chewing_vibration_glasses.png}
    \caption{Vibration Glasses}
    \label{fig:glasses}
\end{figure}

\newpage

In addressing the aforementioned issue, a previous group of researchers
attempted to tackle food classification based on input signals by employing
the Extended Kalman filter in conjunction with additional mounted sensors.
However, this approach necessitated the placement of uncomfortable and
non-portable Electromyography (EMG) sensors on the participant's face.
Consequently, the current project aims to overcome these limitations
by exclusively utilizing captured signals, eliminating the reliance
on EMG data. Deep learning and machine learning techniques, which have
demonstrated promising outcomes in various fields, including signal
processing, will be employed for food classification based on the
signals obtained from the participant's chewing activities.

\section{Data analysis}
Certainly, the quality and reliability of the data play a crucial role in
determining the effectiveness of any model. Therefore, this section seeks
to delve into the attributes of the previously recorded data and to validate
the accuracy of the measurements and assertions made.

In order to provide the readers of this report with a comprehensive
understanding of the data, it is imperative to elucidate the nature
of the signals and the specific circumstances under which the data
was collected.

The experimental cohort consisted of eleven participants who actively engaged in
the study. Each participant wore the specialized glasses and consumed a set
of five distinct food items, namely apple, beef jerky, baguette, cheddar, and chips.
However, it is important to acknowledge that an uneven distribution of data was
observed among the food types. Consequently, certain food items were
not grasped by a few participants, resulting in a scarcity
of data for those specific food categories in comparison to others.

Furthermore, separate signals corresponding to both the food types
and individual participants were obtained and made available for analysis.

The data was acquired using a sampling rate of 24 kHz, resulting in the
collection of 24,000 samples per second.
However, it is important to note that the accuracy of the data acquisition was
not consistent.
As a result, the actual number of samples acquired per second ranged from
24,000 to 24,104, with slight variations observed in different cases.
The following table provides an overview of the participants and the
specific food items they consumed during the data collection process.
\\
\begin{center}
    \begin{tabular}{ |c c c c c c c c c c c|}
        \hline
        \#1        & \#2        & \#3        & \#4   & \#5        & \#6     & \#7      & \#8   & \#9   & \#10       & \#11       \\
        \hline
        apple      & beef jerky & beef jerky & chips & chips      & apple   & chips    & apple & apple & baguette   & apple      \\
        beef jerky & apple      & chips      & chips & chips      & apple   & chips    & apple & chips & baguette   & chips      \\
        apple      &            & chips      & chips & beef jerky & cheddar & chips    &       & chips & baguette   & beef jerky \\
                   &            & chips      & chips & beef jerky & cheddar & chips    &       & chips & apple      & beef jerky \\
                   &            & beef jerky & chips & beef jerky &         & chips    &       & chips & apple      & apple      \\
                   &            & beef jerky & chips & beef jerky &         & chips    &       &       & apple      & chips      \\
                   &            &            &       &            &         & baguette &       &       & apple      & chips      \\
                   &            &            &       &            &         &          &       &       & chips      &            \\
                   &            &            &       &            &         &          &       &       & chips      &            \\
                   &            &            &       &            &         &          &       &       & chips      &            \\
                   &            &            &       &            &         &          &       &       & cheddar    &            \\
                   &            &            &       &            &         &          &       &       & cheddar    &            \\
                   &            &            &       &            &         &          &       &       & beef jerky &            \\
                   &            &            &       &            &         &          &       &       & chips      &            \\
                   &            &            &       &            &         &          &       &       & chips      &            \\
        \hline
    \end{tabular}
\end{center}
\newpage

As evident from the aforementioned table, it is apparent that there is an uneven distribution
of data among the participants.
This uneven distribution poses a challenge for many machine learning and deep learning models,
as they often require a balanced dataset for optimal performance.
However, it is worth noting that various approaches exist to mitigate this issue.


One possible approach is data augmentation, which involves artificially increasing the size of the
dataset by applying transformations such as phase shift, scaling, or adding noise to the existing samples.
This can help balance the dataset and provide additional variations for training the models.

Another approach is to employ techniques such as oversampling or undersampling.
Oversampling involves replicating the minority class samples to match the number
of majority class samples, while undersampling involves randomly removing samples
from the majority class to match the number of minority class samples.
Both techniques aim to achieve a more balanced dataset.

Additionally, ensemble methods, which combine predictions from multiple models,
can be utilized to account for the class imbalance.
By training several models on different subsets of the data or using different algorithms,
ensemble methods can improve the overall performance and robustness of the models.

Overall, while dealing with an uneven distribution of data among participants presents a
challenge, employing techniques such as data augmentation, oversampling, undersampling,
or ensemble methods can help mitigate this problem and enhance the performance
of machine learning and deep learning models.

\subsection{Collected Signals}

Prior to delving into other aspects, it is essential to focus on the
characteristics of the signals themselves and the context in which the
previous team collected the data.
Each participant was seated in a comfortable chair and engaged in a
series of activities, including responding to questions to capture
muscle activity during speech, followed by the consumption of various
food items. Subsequently, the recorded data was partitioned into
multiple CSV files, with each file encompassing approximately 48 MB
of data, corresponding to approximately 1,000,000 records.
The number of CSV files ranged from 20 to 25 for each participant,
depending on the number of foods they consumed.
Within this extensive dataset, the region of interest lies in
the portion pertaining to food chewing.


Additionally, another dataset was compiled specifically for food chewing,
consisting of signals isolated for each food item and participant.
This dataset serves as the training, validation, and testing sets for
the subsequent models. Furthermore, a supplementary dataset comprising
text files was included, containing annotations denoting the onset
of chewing for each chewing sequence within the signal.
Each line in the text files represents a time value in seconds,
signifying the initiation of chewing as mentioned.

\subsection{Verification}
The initial task at hand involves confirming the presence of the chewing
sequence for a specific participant within the complete signal.
To accomplish this, the individual CSV files were combined to construct a
cohesive and comprehensive dataset stored in memory.
Subsequently, given that the recorded signals were susceptible to noise,
a preprocessing technique was employed to enhance the signal quality and
eliminate outliers. Among the various filters tested,
the "hampel" filter was selected for its ability to effectively
remove outliers and achieve signal smoothing.
Notably, it is important to mention that certain
filters, such as the median filter, proved unsuitable
for this particular task as they exhibited a
tendency to distort the input signal.

The Hampel filter is a statistical method commonly used for
outlier detection and removal in signal processing.
It operates by identifying data points that deviate significantly from
the surrounding values and replaces them with more representative estimates.
This filter considers the median absolute deviation (MAD) as a robust measure of
dispersion to detect outliers.
The MAD is calculated by taking the median
of the absolute differences between each data point
and the median of its neighboring values.
The Hampel filter compares each data point to a threshold based
on a specified multiple of the MAD.
If the difference exceeds this threshold, the data point is
considered an outlier and replaced with a more suitable estimate,
such as the median of the surrounding values.

On the other hand, the median filter is a widely used technique
for signal smoothing and noise reduction.
It operates by replacing each data point with the median value within a
defined neighborhood window. The median is less
sensitive to extreme values, making it effective in mitigating the impact
of outliers and preserving the overall shape of the signal.
However, it is worth noting that the median filter can also
introduce certain \bf{distortions} \normalfont in the signal, particularly
when applied to signals with abrupt changes or sharp features.
These distortions may be undesirable in certain applications,
such as the analysis of chewing sequences,
where preserving the fine details of the signal is crucial.

For a comprehensive understanding of the input signal and the impact of
noise, Figure \ref{fig:raw_signal} displays the plotted signal in its raw form.
This visualization provides insights into the inherent
characteristics and fluctuations within the signal.
However, to mitigate the effects of noise and enhance the signal
quality, preprocessing techniques were applied.
Figure \ref{fig:filtered_signal} presents the signal after undergoing the designated preprocessing
steps, showcasing the result of noise reduction and smoothing.
By comparing these two figures, one can gain a better appreciation
of the improvements achieved through the preprocessing stage.

\begin{figure}[!h]
    \centering
    \includegraphics[width=0.75\textwidth]{img/raw_signal.png}
    \caption{Raw signal}
    \label{fig:raw_signal}
\end{figure}

\begin{figure}[!h]
    \centering
    \includegraphics[width=0.75\textwidth]{img/filtered_signal.png}
    \caption{Filtered signal}
    \label{fig:filtered_signal}
\end{figure}
Figure \ref{fig:raw_filtered_signals} showcases a zoomed-in portion
of both the raw signal and the filtered signal.
This magnified view provides a closer examination of the
details within the signal, allowing for a more precise analysis of
the impact of the filtering process.
By visually comparing the raw and filtered versions in this
smaller segment, one can observe the specific
changes and improvements brought about
by the applied filtering technique.

\begin{figure}[!h]%
    \centering
    \subfloat[\centering small portion of raw signal]{{\includegraphics[width=0.4\textwidth]{img/portion_raw_signal.png} }}%
    \qquad
    \subfloat[\centering small portion of filtered signal]{{\includegraphics[width=0.4\textwidth]{img/portion_filtered_signal.png} }}%
    \caption{Filtered and raw signal comparison}%
    \label{fig:raw_filtered_signals}%
\end{figure}

\subsection{Finding masticate signals in entire signal}
It is of utmost importance to validate the previous team's efforts
in accurately determining the chewing sequences within the entire signal.
This verification process serves not only to ensure the correctness and
accuracy of the data but also plays a critical role in establishing
an approach or procedure for identifying such sequences in future work.
By confirming the reliability of the chewing sequence determination,
researchers can establish a robust foundation for subsequent analyses and
investigations related to this specific aspect.

The subsequent section provides a detailed explanation of the chosen approach
for identifying the mentioned sequences.
Cross correlation, a widely accepted method for detecting a sequence
in a signal, was initially considered.
However, extensive research and experimentation revealed that cross correlation
struggles to handle large sequences effectively,
often resulting in confusion regarding the optimal match within
the larger signal.
Cross correlation may face challenges when dealing with large sequences due to
several reasons.
One of the main limitations is the computational complexity associated with
cross correlation calculations for large data sets.
As the sequence length increases,
the number of comparisons required grows exponentially,
resulting in a significant increase in computational resources and time.

Moreover, as the sequence size increases,
the likelihood of finding multiple occurrences
or similar patterns within the larger signal also increases.
This can lead to ambiguity in determining the exact match or
alignment between the two signals, making
it difficult to identify the precise location of the desired sequence.


Furthermore, cross correlation assumes
stationarity and linear relationships between the signals,
which may not hold true in the case of large and complex
data sets. Large sequences often exhibit nonlinear behavior and
non-stationary characteristics, which can affect the accuracy of
cross correlation-based matching.

To address these challenges, alternative approaches such as the
multi-cross correlation method mentioned earlier can be employed.
These modified techniques take into account the specific limitations
of cross correlation for large data sets and offer more efficient
and effective solutions for sequence identification.

To address this limitation, a modified approach called
"multi-cross correlation" was devised.

In the multi-cross correlation approach, a sliding window was employed to
divide the larger signal into segments with 50\% overlap, ensuring
comprehensive coverage of the entire signal. Each window was twice the
length of the small signal portion (the sequence of samples) under
consideration, and regular cross correlation was applied within each window.
Consequently, a match, representing the most probable match sequence,
was identified within each window.

\begin{figure}[!h]
    \centering
    \includegraphics[width=0.75\textwidth]{img/sliding_window.png}
    \caption{Sliding window with 50 per cent overlap}
    \label{fig:sliding_window}
\end{figure}

To determine the correct match, a norm 2 distance calculation and thresholding
technique were utilized. The norm 2 distance measured the dissimilarity between
non-matching signals, with a larger value indicating a greater deviation from
a perfect match.
Note that in the proposed approach, a Fast Fourier Transform (FFT) method has
been utilized to decompose the matched portion of signals within each window.
By applying the FFT, the frequency components of the matched portion can be
analyzed and represented in the frequency domain.
Subsequently, the norm 2 distance calculation is performed on the
decomposed signals.

By computing the norm 2 of the decomposed signals, the dissimilarity between the
matched portion and the larger signal can be quantified.
The norm 2 distance serves as a measure of the overall difference between the
frequency components of the matched portion and the signal itself.
This enables a more comprehensive evaluation of the match and enhances the
accuracy of sequence identification.

The inclusion of the FFT and subsequent norm 2 distance calculation in the approach provides a more refined and detailed analysis of the frequency characteristics of the matched portion within the larger signal. This additional step enhances the effectiveness and reliability of the sequence identification process.
By applying a threshold, the approach accounted for inconsistencies
between the two signals under investigation, allowing for flexibility in accommodating
perturbations and imperfect matches.

\begin{figure}[!h]%
    \centering
    \subfloat[\centering Norm2 not matched FFT]{{\includegraphics[width=0.4\textwidth]{img/fft_not_matching_norm2.png} }}%
    \qquad
    \subfloat[\centering Norm2 matched FFT]{{\includegraphics[width=0.4\textwidth]{img/fft_matching_norm2.png} }}%
    \caption{FFT matching comparison with norm2 distance}%
    \label{fig:fft_signal_matching}%
\end{figure}

The inclusion of a threshold was essential to accommodate variations in the
signals and ensure robustness in identifying the correct match.
This adaptive approach enabled accurate sequence detection despite potential
deviations and inconsistencies between the signals.

Figure \ref{fig:matched_signal} clearly demonstrates the effectiveness of
the proposed algorithm in accurately identifying the desired match within
the entire signal.
The high precision achieved by the algorithm is evident,
as indicated by the close alignment between the identified match and
the expected sequence. This successful outcome validates the robustness
and reliability of the algorithm in accurately locating the
desired sequence within the larger signal.
The results depicted in Figure \ref{fig:matched_signal} provide strong
evidence of the algorithm's ability to
achieve precise and accurate match identification,
further affirming its suitability for the intended purpose.
\begin{figure}[!h]
    \centering
    \includegraphics[width=0.75\textwidth]{img/matched_signal.png}
    \caption{Start of match and matched signal in entire signal}
    \label{fig:matched_signal}
\end{figure}

\subsection{Start of chewing annotation}
Although not directly impacting the modeling section, it is
imperative to validate the information regarding the start
of chewing for each specific type of food within the chewing
sequence dataset. While this validation process may not have a
direct bearing on the modeling aspect, it plays a crucial role
in ensuring the accuracy and reliability of the data.
By verifying the provided information, researchers can ascertain
the correctness of the annotated start times for chewing events,
which in turn enhances the overall quality and integrity of the
dataset.

In the pursuit of identifying the start of chewing within a
sequence of chewing food, a specific approach developed by
a group of researchers was employed.
This approach, implemented by the researchers, successfully enabled
the identification of the precise initiation of
chewing events.

The detection of the beginning of each chew was facilitated by a relatively
simple algorithm, as outlined in the referenced paper.
This algorithm involved evaluating the short-time signal energy within a 20
ms window and comparing it to a predetermined energy threshold.
If the short-time signal energy exceeded the threshold,
the resulting signal was set to 1; otherwise, it was set to 0.
Subsequently, the squared signal was subjected to low-pass
filtering using a 4th order Butterworth filter.
The specific choice of a filter with a 3dB cut-off frequency of
4 to 5 Hz effectively responded to the
pause in phase 4 while filtering out the shorter pause in phase 2.
The hill climbing algorithm was then employed to accurately
detect the beginning of each chew, as depicted in Figure \ref{fig:annotation}.

According to the findings presented in the paper,
this algorithm demonstrated a high success rate,
accurately detecting the start point of approximately 90\% of all
chews. Importantly, the algorithm exhibited minimal false
insertions, further affirming its reliability and effectiveness
in accurately identifying the onset of chewing events. \LaTeX{} \cite{paper1}
\begin{figure}[!h]
    \centering
    \includegraphics[width=1\textwidth]{img/annotation.png}
    \caption{Start of chewing annotation}
    \label{fig:annotation}
\end{figure}
\newpage

The observed differences between the annotated start of chewing
and the start of chewing identified by the algorithm can be attributed
to various factors, including inconsistencies in the definition of
the start of chewing and the manner in which the data was recorded.

The definition of the start of chewing may vary among researchers and
experts, leading to variations in the identification of this specific
event. Different criteria or thresholds may be applied to determine the
precise moment when chewing begins, resulting in discrepancies between
manual annotations and algorithmic detection.

Furthermore, the process of recording the data itself can introduce certain
limitations and challenges. Factors such as the positioning of sensors
, variations in sensor sensitivity, and noise interference can impact
the accuracy of identifying the exact start of chewing.
These factors can contribute to differences between the algorithm's
marked start of chewing and the annotations provided.

It is important to recognize and acknowledge these inconsistencies and limitations when
comparing the algorithm's results with manual annotations.
Such discrepancies can provide insights into the complexities of
accurately identifying the start of chewing and highlight areas
for improvement in future research and algorithm development.

\subsection{Data Preparation for modeling}

Data preparation plays a pivotal role in the modeling process and significantly influences
the ultimate outcomes obtained.
In our study, we have identified several models,
including Support Vector Machines (SVM), Convolutional Neural Networks (CNN)
(some variants including ResNet, EfficientNet, DenseNet),
Recurrent Neural Networks (RNN), Transformers,
and Vision Transformers (ViT), which we aim to implement.
Each of these models has specific requirements and is suitable for
different types of datasets.
Consequently, we have formulated distinct datasets tailored to
meet the specific needs of each model, taking into account their
respective strengths.
To cater to models that excel in sequence-based data analysis, such as
RNNs, we have transformed our input signals into image representations
known as spectrograms. By representing the signals as sequences in the
time domain, we leverage the inherent sequential nature of the data,
enabling effective utilization by these models.
Furthermore, we have explored the extraction of relevant features from the
signals, enabling us to create tabular datasets. This approach is particularly useful
for models that operate on tabular data, facilitating their application and
harnessing the potential insights derived from the transformed signal data.

In the subsequent sections, we will elaborate on the preprocessing techniques employed
and discuss the specific conditions and considerations involved in preparing the
datasets tailored to each model's requirements.

As described in the preceding sections, we have prepared three distinct datasets for the modeling phase.
The first dataset is in a tabular format, composed of statistical features and properties extracted
from our signal in both the time and frequency domains.
This dataset provides a representation of our signal data suitable for models designed
to handle tabular data.

The second dataset consists of time series data, which comprises the original signal segmented
into smaller chunks or subsequences, preserving its temporal sequence. This format caters
to models that specialize in analyzing time series data.

Lastly, we have transformed the smaller time series data into
spectrograms, representing them as images.
This dataset, comprised of spectrogram images, is intended for models
capable of processing visual data.

By creating these three diverse datasets, we can leverage the unique capabilities
of various models and select the most appropriate one for our specific analysis,
allowing for a comprehensive exploration and interpretation of the data from
multiple perspectives.

The approach employed for data preparation involved using the isolated signals of
mastication corresponding to a specific food type. These signals were further segmented
into smaller portions using a sliding window technique with a 60\% overlap and a window
size of approximately 0.8 seconds, resulting in a window size of around 20,000 sample
data points. The choice of 0.8 seconds as the window size was based on the statistical
distribution of chewing cycles within the available data (Figure \ref{fig:time_histogram}).

\begin{figure}[!h]
    \centering
    \includegraphics[width=0.5\textwidth]{img/histogram_time_dist.png}
    \caption{Histogram for one food type based on time distribution}
    \label{fig:time_histogram}
\end{figure}
\newpage

\begin{center}
    \begin{tabular}{ |c|c|c|}
        \hline
        Food Type  & MEAN   & STD    \\
        \hline
        apple      & 0.7498 & 0.0928 \\
        baguette   & 0.8148 & 0.1013 \\
        beef jerky & 0.6631 & 0.1382 \\
        cheddar    & 0.8187 & 0.0661 \\
        chips      & 0.6871 & 0.1192 \\
        \hline
        Total      & 0.7135 & 0.0700 \\
        \hline
    \end{tabular}
\end{center}

In the process of data generation, we employed stratified sampling to ensure that the
distribution of data remained consistent across the different sets—train, validation,
and test. The split distribution for these sets was selected as 75\%, 15\%, and 15\% respectively,
ensuring an appropriate representation of the data in each subset.

Note that, stratified sampling is a sampling technique used in statistics and data analysis to
ensure that each subgroup or category within the dataset is adequately represented in the sample.
The goal of stratified sampling is to create a representative sample that accurately reflects the distribution
of the entire population with respect to specific characteristics or categories.

In the context of data preparation, stratified sampling involves dividing the dataset
into distinct subgroups based on certain attributes or classes.
The samples are then randomly selected from each subgroup in proportion to their representation
in the entire dataset. This approach helps to prevent biased sampling and ensures that
rare or underrepresented categories are not overlooked.

For instance, if we have a dataset with three classes (A, B, and C) and class A represents 60\%
of the data, class B represents 30\%, and class C represents 10\%, stratified sampling
will ensure that the samples chosen for the training, validation, and testing sets will maintain this
distribution.
This helps to produce a balanced and representative sample
that allows for more accurate and reliable analysis and modeling.

By employing this data preparation methodology, we aimed to create representative datasets
that adequately captured the essential characteristics of the chewing signals while maintaining
data integrity and consistency throughout the training, validation, and testing
phases of our analysis.


\subsubsection{Tabular Dataset}
A tabular dataset refers to a structured form of data organization, typically presented in the form
of rows and columns like a table.
Each row represents a specific data sample or instance, while each column corresponds to a particular
attribute or feature of that data sample.
In the context of our study, the tabular dataset is constructed based on statistical features and
properties extracted from the signal data in both the time and frequency domains.

Support Vector Machines (SVM) is a popular machine learning algorithm used for classification
and regression tasks. In the context of classification, SVM aims to divide the data samples into
distinct classes by finding an optimal hyperplane that maximizes the margin between different classes.
This hyperplane serves as the decision boundary, and data points are classified based on which side
of the hyperplane they lie.

To utilize the tabular dataset for classification using SVM, we first define classes based
on the problem at hand. For example, in a binary classification scenario, we might have two classes
: "chew" and "non-chew".
Each row in the tabular dataset represents a data sample, and the columns correspond to
the statistical features extracted from the signal data.

During the training phase, SVM learns the optimal hyperplane that best separates the data samples
into their respective classes. It does this by finding support vectors, which are data points located
closest to the hyperplane. The goal is to maximize the margin between these support vectors while
minimizing the classification error.

Once the SVM model is trained, it can classify new, unseen data samples by determining on which side
of the learned hyperplane they lie. Data samples falling on one side are assigned
to one class (e.g., "chew"), while those on the other side are assigned to the other
class (e.g., "non-chew").

The SVM algorithm's ability to work effectively with tabular datasets lies in its ability to
handle high-dimensional feature spaces and find complex decision boundaries.
By extracting meaningful statistical features from the signal data and constructing a tabular dataset,
we can leverage SVM's strengths for accurate and efficient classification tasks in our study.

In the modeling section, a comprehensive and detailed explanation of
Support Vector Machines (SVM) will be provided.

In this section of the study, we employed a tabular dataset consisting of more than 2500 rows,
each representing a distinct sample. The dataset comprised 16 different features, encompassing
both time domain and frequency domain attributes extracted from the signal data.

The time domain features included:
\begin{enumerate}
    \item Mean
    \item  Standard deviation
    \item  Root Mean Square (RMS)
    \item  Variance
    \item  Skewness
    \item  Kurtosis
    \item  Entropy
\end{enumerate}

The frequency domain features included:
\begin{enumerate}
    \item Mean power
    \item Standard deviation
    \item Spectral centroid
    \item Spectral spread
    \item Spectral skewness
    \item Spectral kurtosis
    \item Band energy ratio
    \item Peak frequency
    \item Spectral entropy
\end{enumerate}


Here is a brief explanation of each feature:
\begin{enumerate}
    \item Mean: The arithmetic average of the data points, representing the central tendency of the distribution.
    \item Standard Deviation (std): A measure of the dispersion or spread of the data points around the mean, indicating the variability of the data.
    \item Root Mean Square (RMS): The square root of the average of the squared data points, capturing the effective magnitude of the signal.
    \item Variance: The average of the squared differences between each data point and the mean, measuring the spread of the data.
    \item Skewness: A measure of the asymmetry of the data distribution, indicating the degree of deviation from a symmetrical bell-shaped curve.
    \item Kurtosis: A measure of the tailedness or heaviness of the data distribution, quantifying the presence of outliers or extreme values.
    \item Entropy: A measure of the uncertainty or disorder in the data, assessing the amount of information contained in the signal.
    \item Mean Power: The average power of the signal in the frequency domain.
    \item Spectral Centroid: The center of mass of the power spectrum, indicating the average frequency of the signal.
    \item  Spectral Spread: A measure of the spread of the power spectrum around its centroid, capturing the bandwidth of the signal.
    \item  Spectral Skewness: A measure of the asymmetry of the power spectrum, indicating the skewness of the frequency distribution.
    \item  Spectral Kurtosis: A measure of the tailedness of the power spectrum, quantifying the presence of extreme values in the frequency distribution.
    \item  Band Energy Ratio: The ratio of the energy in a specific frequency band to the total energy in the signal.
    \item  Peak Frequency: The frequency with the highest amplitude or power in the signal.
    \item  Spectral Entropy: A measure of the amount of information or complexity in the frequency distribution.
\end{enumerate}

Each feature provided valuable insights into the characteristics and properties of the
signal data, capturing relevant information from both the time and frequency domains.
The tabular dataset, equipped with these comprehensive features, served as a robust
foundation for the subsequent implementation of the Support Vector Machines (SVM) algorithm
for classification tasks. By leveraging this feature-rich dataset, we aimed to harness
the discriminative power of SVM to effectively classify the data samples into their
respective classes, thereby addressing the objectives of our study with precision and accuracy.

It is also worth mention that Support Vector Machines (SVM) are versatile and capable of
handling various types of data, including \textbf{image} data.
While SVM is widely known for its effectiveness in handling tabular data and sequence-based data
(time series), it can also be extended to work with image data through appropriate feature
engineering.

When dealing with image data, the key is to represent the images as feature vectors that SVM
can process. This process often involves extracting relevant features from the images, such
as color histograms, texture descriptors, or using techniques like bag-of-visual-words to
represent the images as vectors. Once the images are suitably represented as feature vectors,
SVM can be applied for image classification, object recognition, and other image-related tasks.

By utilizing SVM for image data analysis, we can benefit from its ability to find optimal
decision boundaries in high-dimensional feature spaces, making it suitable for tasks that
involve complex and non-linear relationships in image data. SVM's versatility and effectiveness
in handling multiple data types make it a valuable tool in various machine learning applications
, including those involving image analysis and processing.

\subsubsection{Time Series Dataset}
By implementing the sliding window method for segmenting the signals, we can conveniently
store the resulting small segments into separate files, effectively referring to them
as time series data. This approach allows for the organized and structured storage of the
segmented data, facilitating easy access and analysis of individual time series segments during
subsequent stages of data processing and modeling.

\subsubsection{Spectrogram Dataset}
A spectrogram is a visual representation of the frequency content of a signal over time.
It is a commonly used tool in signal processing and audio analysis to analyze the frequency
components of a time-varying signal, such as audio, speech, or any other time-series data.

\begin{enumerate}
    \item Time Segmentation: To create a spectrogram, the time-varying signal is first divided into short, overlapping segments or frames. Each frame typically ranges from a few milliseconds to tens of milliseconds in duration.
    \item Fourier Transform: For each frame, a mathematical operation known as the Fourier Transform is applied. The Fourier Transform converts the time-domain signal into its corresponding frequency-domain representation. This process decomposes the signal into its individual frequency components, revealing the contribution of each frequency to the signal at that specific moment in time.
    \item Power Spectrum: The magnitude of the Fourier Transform represents the amplitude or strength of each frequency component in the signal. By squaring the magnitude values, we obtain the power spectrum, which quantifies the energy distribution across different frequencies.
    \item Visualization: The power spectrum of each frame is typically displayed as a 2D image, where the x-axis represents time, the y-axis represents frequency, and the color intensity or shading denotes the power or magnitude of the corresponding frequency component. Brighter colors indicate higher power or energy at specific frequencies, while darker colors represent lower power.
\end{enumerate}

The resulting spectrogram provides valuable insights into the frequency content and spectral characteristics
of the time-varying signal.
It allows us to visualize how the signal's frequency components change over time,
identifying patterns, harmonics, and other interesting features that may not be evident
in the time-domain representation alone. Spectrograms are widely used in various fields,
including speech processing, music analysis, radar signal processing, and many other applications
that involve analyzing time-frequency representations of signals.

\begin{figure}[ht]
    \begin{minipage}[b]{0.5\linewidth}
        \centering
        \includegraphics[width=.5\linewidth]{img/spectrogram_1.png}
        \caption{Spectrogram 1}
        \vspace{4ex}
    \end{minipage}%%
    \begin{minipage}[b]{0.5\linewidth}
        \centering
        \includegraphics[width=.5\linewidth]{img/spectrogram_2.png}
        \caption{Spectrogram 2}
        \vspace{4ex}
    \end{minipage}
    \begin{minipage}[b]{0.5\linewidth}
        \centering
        \includegraphics[width=.5\linewidth]{img/spectrogram_3.png}
        \caption{Spectrogram 3}
        \vspace{4ex}
    \end{minipage}%% 
    \begin{minipage}[b]{0.5\linewidth}
        \centering
        \includegraphics[width=.5\linewidth]{img/spectrogram_4.png}
        \caption{Spectrogram 4}
        \vspace{4ex}
    \end{minipage}
    \label{fig:spectrograms}
    \caption{Spectrogram samples from chewing an apple}
\end{figure}
\newpage

\section{Modeling}
\subsection{SVM}
\subsection{RNN}
A Recurrent Neural Network (RNN) is a type of neural network architecture designed to handle sequential data.
Unlike feed-forward neural networks, RNNs introduce feedback loops that allow them to process inputs in a sequential manner,
where the current input not only depends on the current time step but also on the information from previous time steps.
This recurrent structure makes RNNs suitable for tasks involving time series data, natural language processing,
and other sequential data analysis.

The primary challenge with traditional RNNs is the vanishing gradient problem, where gradients become exponentially small during backpropagation through time, leading to difficulties in learning long-range dependencies. As a result, traditional RNNs may struggle to retain information over long sequences, limiting their effectiveness in capturing long-term patterns.
\subsubsection{BiRNN}
Bidirectional RNNs (BiRNNs) address the limitation of traditional RNNs by processing input data in both forward and backward directions.\LaTeX{} \cite{paper2}
By considering the context from both past and future time steps, BiRNNs can capture bidirectional dependencies in the sequential data.
This dual processing of information allows BiRNNs to gain a better understanding of the data and improves their ability to model complex temporal relationships.
\subsubsection{LSTM}
Long Short-Term Memory (LSTM) is a specialized variant of RNNs that overcomes the vanishing gradient problem.\LaTeX{} \cite{paper3}
LSTMs introduce additional memory cells and gating mechanisms, which include an input gate, a forget gate,
and an output gate. These gates control the flow of information in and out of the LSTM cell, allowing the
network to selectively retain or forget information over multiple time steps.

The input gate determines how much of the new input to incorporate into the memory cell, while the forget gate
controls what information to discard from the cell's memory. The output gate governs how much information to
pass to the next time step. The ability to selectively remember and forget information makes LSTMs suitable
for tasks requiring long-term dependencies and memory retention.

Following training with various configurations, ranging from very low learning rates (1e-6)
to finer learning rates (1e-3), and experimenting with increasing the model's capacity
through additional layers and larger hidden states, the unfortunate outcome was that the
train loss reached a value around 1.4. Moreover, as shown in Figure \ref{fig:train_lstm_acc_dsv2},
the train accuracy did not surpass approximately 36 percent,
indicating that the model did not effectively learn meaningful patterns from the data.

\begin{figure}[!h]
    \centering
    \includegraphics[width=0.5\textwidth]{img/train_loss_lstm_datasetv2.png}
    \caption{Train accuracy for LSTM on dataset v2.0}
    \label{fig:train_lstm_acc_dsv2}
\end{figure}

The limited performance can be attributed to several factors, including:
\begin{enumerate}
    \item A potential bug in the implementation.
    \item Poor quality of the data used for training.
    \item Insufficient model capacity, which may hinder its ability to capture complex patterns.
    \item Low learning rate, which might slow down the learning process and hinder convergence.
\end{enumerate}

After careful debugging and ruling out the presence of a bug, as well as addressing the issues of
low model capacity and learning rate, the most likely remaining cause for the suboptimal performance
is the quality of the data used to train the model. Improving the data quality might lead to better
results and allow the model to learn more meaningful representations from the available data.

As shown in Figure \ref{fig:signal_vis_dsv2}, it is evident that signals with the same label
exhibit a highly diverse range of shapes and patterns.
This substantial variation among signals with identical labels could potentially be the
underlying cause for the model's confusion and subsequent low accuracy.
The diverse shapes in signals belonging to the same class may introduce significant challenges
for the model during the learning process.
It becomes difficult for the model to generalize effectively and distinguish between
different instances of the same class due to the substantial intra-class variations.

\begin{figure}[ht]
    \begin{minipage}[b]{0.5\linewidth}
        \centering
        \includegraphics[width=.75\linewidth]{img/lstm_data_vis_l2.png}
        \caption{Label 2 (beef jerky)}
        \vspace{4ex}
    \end{minipage}%%
    \begin{minipage}[b]{0.5\linewidth}
        \centering
        \includegraphics[width=.75\linewidth]{img/lstm_data_vis_l4.png}
        \caption{Label 4 (chips)}
        \vspace{4ex}
    \end{minipage}
    \caption{Signals from two specific labels}
    \label{fig:signal_vis_dsv2}
\end{figure}

In such cases, the model may struggle to learn representative and discriminative features
for each class, resulting in misclassifications and reduced accuracy.
To address this issue, efforts should be made to enhance the quality of the training data
, ensure consistency in data collection, and potentially explore data augmentation techniques
to increase the diversity of samples within each class. These measures can aid
the model in better capturing the underlying patterns and improve its accuracy when
dealing with signals exhibiting considerable intra-class variations.

\subsubsection{New Version of the Data Set}
A promising and effective approach to address the challenges posed by the diverse shapes
and characteristics within the same class of signals is to split the data based on annotations.
By considering both the food signals and the corresponding chewing annotations, a newer
version of the dataset has been created, specifically tailored for RNN and Transformer
models.

The advantage of this new dataset lies in its ability to accommodate different data
lengths, which is essential for RNN and Transformer architectures that can handle
variable-length sequences. By associating the food signals with their corresponding
chewing annotations, the dataset aims to establish a higher level of similarity among
signals with similar labels in terms of their shapes and other characteristics.

This approach seeks to provide the models with more coherent and consistent examples
for each class, reducing the intra-class variations and enabling them to better
capture the underlying patterns and similarities. As a result, the RNN and
Transformer models can benefit from a more focused and representative dataset,
potentially leading to improved accuracy and performance in classifying signals
with diverse shapes but similar labels.

\subsubsection{GRU}
The Gated Recurrent Unit (GRU) \LaTeX{} \cite{paper4} is another variant of RNNs that shares some similarities with LSTM but has a simpler architecture.
GRUs use a reset gate and an update gate, combining the functionality of the input and forget gates in LSTM.
The reset gate controls which past information to forget, while the update gate determines how much of the
new input to incorporate into the cell's memory. GRUs have been shown to be computationally more efficient than
LSTMs, making them particularly suitable for smaller datasets and applications requiring medium-range dependencies.

In summary, RNNs and their variants like BiRNN, LSTM,
and GRU are powerful tools for processing sequential data.
Each variant has specific strengths and applications, and researchers continue to
explore improvements and variations to better address the challenges of capturing temporal
dependencies and long-term information in sequential data.

\subsection{Transformers}
The Transformer is a deep learning architecture that was initially introduced for natural language
processing tasks, particularly machine translation. It has since become a fundamental building block
in various applications due to its ability to handle sequential data efficiently.\LaTeX{} \cite{paper5}

In the context of time domain signal classification tasks, the Transformer can be adapted
to process the sequential data effectively. Instead of using traditional RNN-based approaches,
the Transformer employs self-attention mechanisms to capture long-range dependencies within
the time domain signal.

Self-attention allows the model to weigh the importance of different time steps when processing a
particular time step. This attention mechanism enables the Transformer to consider the entire signal
simultaneously, making it well-suited for capturing temporal relationships and dependencies,
even across long sequences.

In the classification task, the Transformer can be combined with additional layers to perform
the final classification based on the extracted features from the time domain signal.
The model can efficiently process variable-length signals without the need for recurrent
connections, leading to faster training and reduced computational complexity.

Overall, the Transformer's attention-based architecture makes it a powerful choice for time
domain signal classification tasks, allowing it to effectively capture temporal patterns
and dependencies in the data, leading to accurate and efficient classification results.
\subsection{CNN}
\begin{thebibliography}{9}
    \bibitem{paper1}
    Amft, O., et al. "Analysis of Chewing Sounds for Dietary Monitoring." section. 5.1,  pp. [12].
    \bibitem{paper2}
    Schuster, M., \& Paliwal, K. K. (1997). Bidirectional recurrent neural networks. IEEE Transactions on Signal Processing, 45(11), 2673-2681.
    \bibitem{paper3}
    Hochreiter, S., \& Schmidhuber, J. (1997). Long short-term memory. Neural computation, 9(8), 1735-1780.
    \bibitem{paper4}
    Cho, K., Van Merriënboer, B., Gulcehre, C., Bahdanau, D., Bougares, F., Schwenk, H., \& Bengio, Y. (2014). Learning phrase representations using RNN encoder-decoder for statistical machine translation. arXiv preprint arXiv:1406.1078.
    \bibitem{paper5}
    Attention Is All You Need by Vaswani et al. (2017)
\end{thebibliography}
% ---------- End of Document ----------
\end{document}
